\documentclass[a4paper, 11pt]{article}
\usepackage[utf8]{inputenc}
\usepackage[T1]{fontenc}
\usepackage{lmodern}
\usepackage{titlesec}
\usepackage{amsfonts}
\usepackage{amsmath}
\usepackage{color}
\usepackage[txtcentered=true, height=40pt, width=70pt]{thumbs}
\usepackage[left=1.5cm, top=2cm, right=2.5cm]{geometry}
\usepackage{subcaption}
\usepackage{circuitikz}
\usepackage{trfsigns}
\usepackage{pgfplots}
\usepackage{textcomp}
\usepackage{multicol}
\usepackage{upgreek}
\usepackage{empheq}
\usepackage{array}
\usepackage{float}

\pagenumbering{arabic}

\newcommand{\fancythumb}[2]{
	\addthumb{#1}{\large\sffamily\textbf{\space\space#1\vspace{5pt}}}{white}{#2}
}

\newcommand{\fancyformula}[2]{
        \small
        \raggedright{\sffamily\textbf{#1}}
        #2
}

\newcommand{\ftransform}{~\laplace~}

\usepgfplotslibrary{fillbetween}

\DeclareMathOperator{\Si}{Si}
\DeclareMathOperator{\sinc}{sinc}
\DeclareMathOperator{\sgn}{sgn}
\DeclareMathOperator{\rect}{rect}
\DeclareMathOperator{\ld}{ld}

\titleformat*{\section}{\sffamily\Large\bfseries}
\titleformat*{\subsection}{\sffamily\large\bfseries}
\titleformat*{\subsubsection}{\sffamily\normalsize\bfseries}
\titleformat*{\paragraph}{\sffamily\normalsize\bfseries}

\begin{document}

\section*{Abstastung und Quantisierung}
\fancyformula{Abtasttheorem}{
	\[
		f_s \geq 2 ~ f_{\mathrm{max}} = f_{s, N}
	\]
}

\fancyformula{Nyquist-Frequenz}{
	\[
		f_N = \frac{1}{2} ~ f_s
	\]
}

\fancyformula{Nyquist-Rate}{
	\[
		f_{s, N} = 2 ~ f_{\mathrm{max}}
	\]
}

% TODO: Rekonstruktions-Tiefpass-Filter

\section*{Erstes Nyquist-Kriterium}
Bei der Schrittgeschwindigkeit $R_s = \frac{1}{T_s}$ erfüllt ein Impuls $g(t)$ das erste Nyquist-Kriterium, wenn
\[
	g(t) \overset{!}{=} \begin{cases}
		1 & t = 0\\
		0 & t = k ~ T_s, ~ k \neq 0, ~ k \in \mathbb Z
	\end{cases} \qquad \text{(Zeitbereich)}
\]
\[
	\Leftrightarrow
\]
\[
	\frac{\omega_s}{2 \pi} ~ \sum_{k = -\infty}^\infty G(\omega - k \omega_s) \overset{!}{=} 1 \quad \text{mit} ~ \omega_s = \frac{2 \pi}{T_s} \qquad \text{(Frequenzbereich)}
\]

Frequenzbereich intuitiv: Um $\ldots, ~ -\omega_s, ~ 0, ~ \omega_s, ~ 2 \omega_s, ~ \ldots$ verschobene Spektra addieren sich zu $1$ auf!
Wenn das erste Nyquist-Kriterium erfüllt ist, dann gibt es keine Intersymbol-Interferenz (ISI).

\textbf{Hinweis:} Das zweite Nyquist-Kriterium ist heute kaum noch praxisrelevant.

\section*{Raised-Cosine-Impulsformung}
$\alpha$ \ldots ``roll-off''-Faktor, $T = \frac{1}{R_s}$ mit $R_s$ \ldots Symbolrate. Für $\alpha = 0$ ist der Raised-Cosine-Impuls ein $\sinc$-Impuls, d.h. ein idealer Tiefpass.

\begin{multicols}{2}
	\fancyformula{Impulsantwort $g(t)$}{
		\[
			g(t) ~ T= \begin{cases}
				1 & t = 0 \\
				\frac{\sin(\pi / (2 \alpha))}{\pi / (2 \alpha)} ~ \frac{\pi}{4} & |t| = \frac{T}{2 \alpha} \\
				\frac{\sin(\pi t / T)}{\pi t / T} ~ \frac{\cos(\alpha \pi t / T)}{1 - (2 \alpha t / T)^2} & \text{sonst}
			\end{cases}
		\]
	}

	\fancyformula{Basisband-Bandbreite}{
		\[
			B_{\mathrm{BB}} = \frac{1}{2} ~ (1 + \alpha) ~ R_s
		\]
	}

	\fancyformula{Spektrum $G(f)$}{
		\[
			G(f) = \begin{cases}
				1 & |f| \leq \frac{1 - \alpha}{2T} \\
				\frac{1}{2} ~ \left(1 + \cos \left(\frac{\pi T}{\alpha} ~ (|f| - \frac{1 - \alpha}{2 T}) \right) \right) & \frac{1 - \alpha}{2 T} \leq |f| \leq \frac{1 + \alpha}{2 T} \\
				0 & \text{sonst}
			\end{cases}
		\]
	}

	\fancyformula{Bandpassbereich-Bandbreite}{
		\[
			B_{\mathrm{BP}} = (1 + \alpha) ~ R_s
		\]
	}
\end{multicols}

% TODO: Möglicherweise Zeichnung von G(f)?

\section*{Gray-Labelling}
\section*{Rauschen und Symbolfehlerwahrscheinlichkeit}
\subsection*{Allgemein}
\begin{figure}[H]
\centering
\begin{subfigure}{0.54\textwidth}
	\[
		p_n(\xi) = \frac{1}{\sqrt{2 \pi} \sigma} ~ e^{-\frac{1}{2\sigma^2} ~ \xi^2}
	\]
	\[
		P[n \leq c] = \int_{-\infty}^c p_n(\xi) ~ \mathrm d\xi = \mathrm{CDF}\left( \frac{c}{\sigma} \right) = \Phi \left( \frac{c}{\sigma} \right)
	\]
	\[
		P[n > c] = \int_c^\infty p_n(\xi) ~ \mathrm d\xi = \mathrm{CCDF}\left( \frac{c}{\sigma} \right) = Q \left( \frac{c}{\sigma} \right)
	\]
	\[
		\Phi(x) = Q(-x)
	\]
	\[
		Q(x) = 1 - \Phi(x)
	\]
	\[
		\implies \Phi(x) = 1 - \Phi(-x), \quad Q(x) = 1 - Q(-x)
	\]
\end{subfigure}
\begin{subfigure}{0.45\textwidth}
	\caption*{Wahrscheinlichkeitsdichte für $n \sim \mathcal{N}(0, \sigma^2)$}
	\scalebox{0.8}{
	\begin{tikzpicture}
		\begin{axis}[ 
			xlabel = {$\xi$},
			ylabel = {$p_n(\xi)$},
			xmin = -2,
			xmax = 2,
			ymin = -0.2,
			samples = 200
		]

			\addplot[name path=f, green!60!black, mark=none]{1 / (sqrt(0.25 * 2 * pi)) * (e^(-x^2 / (2 * 0.25)))};
			\addplot[name path=axis, black, no markers] coordinates {(-2,0) (2,0)};
			\addplot [mark=none, thick, red!75!black] coordinates {(0.3, -0.05) (0.3, 0.666)};
			\addplot[
				fill=green!60!black, 
				fill opacity=0.2
			]
			fill between[
				of=f and axis,
				soft clip={domain=0.3:2},
			];

			\addplot[
				fill=blue!65!black, 
				fill opacity=0.2
			]
			fill between[
				of=f and axis,
				soft clip={domain=-2:0.3},
			];

		    \node at (axis cs:  0.65, 0.1) {$Q (\frac{c}{\sigma})$};
		    \node at (axis cs:  -0.65, 0.1) {$\Phi(\frac{c}{\sigma})$};
   		    \node at (axis cs:  0.3, -0.1) {\footnotesize{\textit{\color{red!75!black} c}}};
		\end{axis}
	\end{tikzpicture}}
\end{subfigure}
\end{figure}

\subsection*{Für verschiedene Modulationsarten}
Mit Basisbandrauschleistung $N_0 = 2 \sigma_n^2$ ergibt sich Basisband-SNR zu $\mathrm{SNR}_s = \frac{E_s}{N_0}$.

\begin{tabular}{r | c c}
	& SER, $P_{s, \mathrm{err}}$ & BER, $P_{b, \mathrm{err}}$, * \\ \hline \\[-1.0em]
	BPSK & $Q \left( \sqrt{2 ~ \mathrm{SNR}_s} \right)$ & $Q \left( \sqrt{2 ~ \mathrm{SNR}_b} \right)$ \\
	QPSK & $\sim 2 ~ Q \left( \sqrt{\mathrm{SNR}_s} \right)$  & $Q \left( \sqrt{2 ~ \mathrm{SNR}_b} \right)$ \\
	M-PSK & $\sim 2 ~ Q \left( \sqrt{2 ~ \mathrm{SNR}_s} ~ \sin(\pi / M) \right)$ & $\sim \frac{2}{\ld(M)} ~ Q \left( \sqrt{2 ~ \mathrm{SNR}_b ~ \ld(M)} ~ \sin(\pi / M) \right)$ \\
	M-PAM & $2 ~ \left( 1 - \frac{1}{M} \right) ~ Q \left(\sqrt{\frac{6}{M^2 - 1} ~ \mathrm{SNR}_s} \right)$ & $~\sim \frac{2}{\ld(M)} ~ \left( 1 - \frac{1}{M} \right) ~ Q \left(\sqrt{\frac{6 ~ \ld(M)}{M^2 - 1} ~ \mathrm{SNR}_b} \right)$ \\
	M-QAM & $\sim 4 ~ Q \left( \sqrt{\frac{3 ~ \mathrm{SNR}_s}{M - 1}} \right)$ & $\sim \frac{4}{\ld(M)} ~ Q \left( \sqrt{\frac{3 ~ \mathrm{SNR}_b ~ \ld(M)}{M - 1}} \right)$
\end{tabular}

\vspace{5pt}

* Für Coderate $R_c = 1$ (keine Redundanzcodierung) folgt $\mathrm{SNR}_b = \frac{\mathrm{SNR}_s}{M_b} = \frac{\mathrm{SNR}_s}{\ld(M)}$.

Näherung für hohes SNR und Gray-Labeling:
\[
	P_{s, \mathrm{err}} = M_b ~ P_{b, \mathrm{err}}
\]

\section*{Partial-Response-Impulsformung}
\section*{Puls-Amplituden Modulation (PAM)}
\begin{figure}[H]
	\begin{subfigure}{0.54\textwidth}
		\[
			a_{k} \in \{-(M - 1) + 2l ~ | ~ l = 0, \ldots, M-1\}
		\]

		Für Normierung $P_S = E \left[|\tilde{a_k}|^2 \right] \overset{!}{=} 1$ wähle
		\[
			\tilde a_k = \mathrm{norm} \cdot a_k \quad \text{mit} \quad \mathrm{norm} = \frac{1}{\sqrt{\frac{M^2 - 1}{3}}}
		\]
	\end{subfigure}
	\begin{subfigure}{0.45\textwidth}
		\textbf{TODO}:Bild einfügen 
	\end{subfigure}
\end{figure}


\section*{Einseitenband-AM und Hilbertfilter}
\section*{QAM}
Der \textit{\textbf{constellation mapper}} ordnet den Symbolen $M$ verschiedene I/Q-Werte (Konstellationen) zu. Pro Symbol werden also
\[
	M_b = \ld M ~ \left[ \frac{\mathrm{bit}}{\mathrm{symbol}} \right]
\]

Bits übertragen. Mit einer \textbf{Symbolrate} (= Schrittgeschwindigkeit) $R_s ~ \left[\frac{\mathrm{symbol}}{s} = \mathrm{Baud}\right]$ ergibt sich die \textbf{Bitrate} $R_b$ zu
\[
	R_b = M_b ~ R_s \quad \left[ \frac{\mathrm{bit}}{s} \right]
\]

\section*{Sender- / Empfängerunzulänglichkeiten}
\subsection*{Frequenzoffset}
Für jedes Symbol steigt der Phasenoffset um
\[
	\varphi_{\mathrm{inc}} = 2 \pi f_{\mathrm{off}} ~ T_s
\]

\section*{OFDM und FFT / iFFT}
Schrittgeschwindigkeit $R_s = \frac{1}{T_s}$, Abtastrate $R_{\mathrm{samp}}$, Bitrate $R_b$, Unterträgerabstand $\Delta f$, Bandbreite $B$:
\[
	R_{\mathrm{samp}} = N_{\mathrm{sub}} ~ R_s = B
\]
\[
	R_b = M_b ~ N_{\mathrm{sub}} ~ R_s
\]
\[
	\Delta f = R_s
\]

% TODO: FFT / iFFT-Matrix

\section*{Mathe-Formelsammlung}
\subsection*{Additionstheoreme}
\[ \sin(\alpha) ~ \cos(\beta) = \frac{1}{2} (\sin(\alpha - \beta) + \sin(\alpha + \beta)) \]
\[ \cos(\alpha) ~ \cos(\beta) = \frac{1}{2} (\cos(\alpha - \beta) + \cos(\alpha + \beta)) \]
\[ \sin(\alpha) ~ \sin(\beta) = \frac{1}{2} (\cos(\alpha - \beta) - \cos(\alpha + \beta)) \]
\vspace{0.5pt}
\[ \sin(\alpha + \beta) = \sin(\alpha) ~ \cos(\beta) + \cos(\alpha) ~ \sin(\beta) \]
\[ \sin(\alpha - \beta) = \sin(\alpha) ~ \cos(\beta) - \cos(\alpha) ~ \sin(\beta) \]
\[ \cos(\alpha + \beta) = \cos(\alpha) ~ \cos(\beta) - \sin(\alpha) ~ \sin(\beta) \]
\[ \cos(\alpha - \beta) = \cos(\alpha) ~ \cos(\beta) + \sin(\alpha) ~ \sin(\beta) \]
\vspace{0.5pt}
\[ \sin^2(\alpha) + \cos^2(\alpha) = 1 \]
\[ \sin^2(\alpha) - \cos^2(\alpha) = -\cos(2\alpha) \]
\[ \cos^2(\alpha) - \sin^2(\alpha) = \cos(2\alpha) \]
\[ \cos(x)^2 = \frac{1}{2} \left(1 + \cos(2x) \right) \qquad \sin(x)^2 = \frac{1}{2} \left(1 - \cos(2x) \right) \]

\subsubsection*{Wichtige Eigenschaften der Fouriertransformation}
\begin{multicols}{2}
	\fancyformula{Spiegelung}{\[ x(-t) \ftransform X(-\omega) \]}
	\fancyformula{Konjugiert komplex}{\[ x^*(t) \ftransform X^*(-\omega)\]}
	\fancyformula{Verschiebung}{
		\begin{align*}
			x(t - t_0) & \ftransform e^{-j \omega t_0} ~ X(\omega) \\
			e^{j \omega_0 t} ~ x(t) & \ftransform X(\omega - \omega_0)
		\end{align*}
	}	
	\fancyformula{Anfangswert}{
		\[ X(0) = \int_{-\infty}^{\infty} x(t) ~ \mathrm dt \qquad x(0) = \int_{-\infty}^{\infty} X(\omega) ~ \frac{\mathrm d\omega}{2 \pi} \]	
	}
\end{multicols}


\subsubsection*{Wichtige Fouriertransformationen}
Sei Definition der Rechteckfunktion:
\[
	\rect(t) = \begin{cases}
		1 & \text{für } |t| < 1\\
		0 & \text{für } |t| > 1\\
	\end{cases}
\]

Seien $T > 0$, $\omega_0 > 0$:
\begin{multicols}{2}
	\[ \rect \left(\frac{t}{T} \right) \ftransform 2 T ~ \sinc(\omega T) \]
	\[ \frac{\omega_0}{\pi} ~ \sinc(\omega_0 t) \ftransform \rect \left( \frac{\omega}{\omega_0} \right) \]
	\[ \cos(\omega_0 t - \varphi) \ftransform \pi ~ [\delta(\omega - \omega_0) ~ e^{-j \varphi} + \delta(\omega + \omega_0) ~ e^{j \varphi}] \]
	\[ \sin(\omega_0 t - \varphi) \ftransform \frac{\pi}{j} ~ [\delta(\omega - \omega_0) ~ e^{-j \varphi} - \delta(\omega + \omega_0) ~ e^{j \varphi}] \]
	\[ x(t) ~ \cos(\omega_0 t - \varphi) \ftransform \frac{1}{2} ~ \left[X(\omega - \omega_0) ~ e^{-j \varphi} + X(\omega + \omega_0) ~ e^{j \varphi} \right]\]
	\[ x(t) ~ \sin(\omega_0 t - \varphi) \ftransform \frac{1}{2j} ~ \left[X(\omega - \omega_0) ~ e^{-j \varphi} - X(\omega + \omega_0) ~ e^{j \varphi} \right]\]
	\[ \frac{1}{\pi t} \ftransform - j \sgn(\omega)\]
\end{multicols}


% TODO: Anwendung der Q-Funktion irgendwie graphisch darstellen, also was bedeutet Q((D - a) / \sigma) bildlich?
% TODO: Eigenschaften der Q-Funktion und PHI-Funktion
% TODO: Mittlere signalleistung: E_s = E[|x|^2]
% TODO: Definition raised cosine-Impuls, Bandbreite
% TODO: Normierungsfaktor für M-PAM

\end{document}

