\documentclass[a4paper, 11pt]{article}
\usepackage[utf8]{inputenc}
\usepackage[T1]{fontenc}
\usepackage{lmodern}
\usepackage{titlesec}
\usepackage{amsmath}
\usepackage{color}
\usepackage[txtcentered=true, height=40pt, width=70pt]{thumbs}
\usepackage[left=1.5cm, top=2cm, right=2.5cm]{geometry}
\usepackage{subcaption}
\usepackage{circuitikz}
\usepackage{textcomp}
\usepackage{multicol}
\usepackage{upgreek}
\usepackage{empheq}
\usepackage{array}
\usepackage{float}

\pagenumbering{arabic}

\newcommand{\fancythumb}[2]{
	\addthumb{#1}{\large\sffamily\textbf{\space\space#1\vspace{5pt}}}{white}{#2}
}

\newcommand{\fancyformula}[2]{
        \small
        \raggedright{\sffamily\textbf{#1}}
        #2
}

\DeclareMathOperator{\Si}{Si}

\titleformat*{\section}{\sffamily\Large\bfseries}
\titleformat*{\subsection}{\sffamily\large\bfseries}
\titleformat*{\subsubsection}{\sffamily\normalsize\bfseries}
\titleformat*{\paragraph}{\sffamily\normalsize\bfseries}

\begin{document}

\section*{Abstastung und Quantisierung}
\fancyformula{Abtasttheorem}{
	\[
		f_s \geq 2 ~ f_{\mathrm{max}} = f_{s, N}
	\]
}

\fancyformula{Nyquist-Frequenz}{
	\[
		f_N = \frac{1}{2} ~ f_s
	\]
}

\fancyformula{Nyquist-Rate}{
	\[
		f_{s, N} = 2 ~ f_{\mathrm{max}}
	\]
}

\section*{Nyquist-Kriterien}
\section*{Gray-Labelling}
\section*{Rauschen und Symbolfehlerwahrscheinlichkeit}
\section*{Partial-Response-Impulsformung}
\section*{Einseitenband-AM und Hilbertfilter}
\section*{QAM}
\section*{Sender- / Empfängerunzulänglichkeiten}
\section*{OFDM und FFT / iFFT}
\section*{Mathe-Formelsammlung}

% TODO: Anwendung der Q-Funktion irgendwie graphisch darstellen, also was bedeutet Q((D - a) / \sigma) bildlich?
% TODO: Eigenschaften der Q-Funktion und PHI-Funktion
% TODO: Mittlere signalleistung: E_s = E[|x|^2]

\end{document}

